\documentclass{beamer}

\usetheme{AnnArbor}

\usepackage[english]{babel}
\usepackage[utf8x]{inputenc}
\usepackage{url}

\author[Stéphane Bortzmeyer]
{
  Stéphane Bortzmeyer, AFNIC \texttt{bortzmeyer+ietf@nic.fr}
}

\title[Hackathon]{Hackathon Experiences}

\date[IETF 101]
{IETF 101, London, 22 march 2018}

\begin{document}

\begin{frame}
\titlepage  
\end{frame}

\begin{frame}
  \frametitle{Data}
  \begin{itemize}
  \item Six persons working on DoH
  \item We have five public DoH servers (three ``production''), all
    using different code bases
  \item Four (available) server software (Python, Lua…)
  \item Five client software (Python, C, JavaScript, Perl…)
  \item Two libraries  
  \end{itemize}
  \url{https://github.com/IETF-Hackathon/ietf101-project-presentations/tree/master/DoH}
\end{frame}

\begin{frame}
  \frametitle{Results}
  \begin{itemize}
  \item Everyone happily talks to each other
  \item No surprise: DoH is a no-brainer
  \item Running over HTTPS challenges some assumptions in the libraries   
  \item Some things are not fully specified  
  \end{itemize}
\end{frame}

\begin{frame}
  \frametitle{Issues seen during the hackathon}
  \begin{itemize}
  \item GET, POST, OPTIONS and HEAD (risk for interoperability)
  \item Only one server returns Expires: Should it? (``Caching model''
    thread in october 2017)
  \item ``Translation'' of DNS response in HTTP status code and/or headers  
  \end{itemize}
\end{frame}

\end{document}
